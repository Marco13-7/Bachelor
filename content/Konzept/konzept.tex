\chapter{Konzept}

\section{Neue Architektur}
Die neue Architektur soll eine gute und stabile Basis für alle drei Anwendungen bieten. Das soll durch das Erstellen von modularen Komponenten geschehen. Der Kernstruktur der drei Anwendungen soll identisch sein, sodass möglichst wenig unterschiedlicher Code notwendig ist, um die verschiedenen Anwendungen zu erstellen. Bei den Softwaremodulen ist es wichtig, dass diese eine eindeutige Funktion haben, dass eindeutige Schnittstelen definiert sind und dass diese gut wiederverwendet werden können. 

\section{SimulatorMain}
Die Struktur des \texttt{Radarsimulators}, wird im Kern beibehalten. Diese Struktur beinhaltet eine \texttt{SimulatorMain}-Klasse mit einem \texttt{RestartSevice}, eine \texttt{SimulatorInstance}, welche die Funktionen und Verbindungen zur Verfügung stellt und eine View-komponente, welche für die Benutzeroberfläche zuständig ist.  Die \texttt{SimulatorInstance} wird über das Interface \texttt{ISimulator} bedient, welches in Abbilung \ref{fig:SimulatorInstance} zu sehen ist. Die View-Komponente wurde so aufgebaut, dass jede der drei Anwednung diese verwenden kann. Im Kapitel 4.3 wird gezeigt, wie das funktioniert. Der \texttt{RestartService} konnte nicht für jede einzelne Anwendung verallgemeinert werden, da die Instancen verschiedene Startparameter benötigen. Deshalb gibt es eine \texttt{AbstractSimulatorMain}-Klasse, von der die Main-Klassen der weiteren Anwendungen erben. In Abbildung \ref{fig:abstractMain} ist der genaue Aufbau zu erkennen. 

\begin{figure}[ht]
    \centering
    \includegraphics[width=0.9\textwidth]{content/assets/Kapitel4/AbstractMain.png}
    \caption{\texttt{AbstractSimulatorMain} und erbende Klassen}
    \label{fig:abstractMain}
\end{figure}

Um diese Kernfunktionen den Simulatoren zur Verfügung zu stellen gibt es ein Modul namens \texttt{SimulatorMainCore}. Dieses beinhaltet die AbstractSimulatorMain und weitere Klassen, die eine Eclipse-RCP Applikation benötigt.

\subsection{Benutzeroberfläche}

Eine Komponente der Anwendung ist die Benutzeroberfläche. Wie bereits in der Analyse festgestellt wurde, unterscheidet sich die Benutzeroberfläche zwischen Anwendungen durch die Zusammensetzung der Tabs. Es soll vermieden werden, dass jeder Tab einzeln deklariert und initialisiert wird. Denn dadurch entstehen mehrere feste Anhängigkeiten und redundanter Code, was zu schwer wartbarem Code führt. Der neue Ansatz ist nun, dass man für jeden einzelnen Tab, ein View-Modul erstellt. Die Aufgabe eines View-Moduls ist es Funktionen und den Inhalt des Tab Fensters zur Verfügung zu stellen. Mit Hilfe von Extension Points können alle verfügbaren View-Module, innerhalb des Anwendungskontextes geladen und der Benutzeroberfläche hinzugefügt werden.


\begin{figure}[ht]
    \centering
    \includegraphics[width=0.7\textwidth]{content/assets/Kapitel4/ExtensionPointEasy.png}
    \caption{View Extenison Point}

\end{figure}

Jeder Tab-Controller wird in ein eigenes Plug-In ausgelagert, welches einen speziellen ExtensionPoint bereitstellt. Der ExtensionPoint besteht in diesem Fall aus einer Java-Klasse und einem Integerwert. Die im Extension Point angegebene Java-Klasse muss ein festgelegtes Interface implementieren. Der Integer Wert gibt die Reihenfolge an in der die Tabs angeordnet werden sollen.

\begin{figure}[ht]
    \centering
    \includegraphics[width=1\textwidth]{content/assets/Kapitel4/konzeptBenutzeroberfläche1.png}
    \caption{View Extenison Point}
\end{figure}


Mit Hilfe eines \texttt{ExtensionPointLoader}, werden alle verfügbaren Extension Points geladen. Diese liefern ein Objekt der angegebenen Klasse und einen Integer Wert, der die Priorität des Tabs angibt. Nachdem alle ExtensionPoints geladen wurden, werden dessen Objekte und Integer-Werte in einer Liste gespeichert. Daraufhin werden die Nodes, welche von den Objekten geliefert wurden, zu einer weiteren Liste hinzugefügt und nach Priorität sortiert. Diese Liste wird an den \texttt{SimulatorMainController} übergeben. Dieser Iteriert über jedes Element der Liste und fügt die Nodes der Benutzeroberfläche hinzu. 

Der Anwendungskontext wird für jede Anwendung einzeln definiert. Durch dieses Prinzip lassen sich die Tabs der Anwendungen konfigurieren, indem man die passenden Module dem Anwendungskontext hinzufügt. Wie das genau funktioniert wird in Kapitel 5.X gezeigt.

Vorteile:
\begin{itemize}
    \item Kein redundanter Code.
    \item Code ist strukturiert.
    \item Die Benutzeroberfläche kann einfach angepasst werden
\end{itemize}

Nachteile
\begin{itemize}
    \item Kann für Entwickler ohne Extension Point Erfahrung unübersichtlich werden
    \item Es können viele Module entstehen    
\end{itemize}

Was bei der neuen Benutzeroberfläche auch zu beachten ist, dass der \texttt{Testsimulator} für jeden Sensor eine unterschiedliche Benutzeroberfläche hat. Z.B. der AlerterTab wird nur initialisiert, wenn der Sensor ein GA10 ist. Bei jedem anderen Sensor wird der Tab gelöscht. Dieses Problem wurde im alten \texttt{SimulatorMainController} mit einer 70 Zeilen langen Logik, die aus switch-case und if-bedingungen bestand, gelöst. Durch die neue Struktur wird diese Logik von den einzelnen Modulen übernommen. Diese überprüfen je nach Tab den Sensortyp und die Asterix-Version und entscheiden, ob der Tab hinzugefügt werden soll. Wenn der Tab verwendet werden soll geben sie den Node des Panes zurück, wenn nicht geben sie \texttt{null} zurück. Der Extension Point Loader filtert alle Objekte heraus, deren Node \texttt{null} entspricht.


\newpage
\section{Simulator Instance}
Die \texttt{SimulatorInstance} unterscheidet sich bei jeder Anwendungsausprägung, aufgrund der verschiedenen Zusammensetzung von Softwarekomponenten, wie z.B. Netzwerkschnittstellen oder Tasks. Jedoch gibt es Komponenten, welche von mehreren der Anwendungen verwendet werden können. Im Folgenden wird die Grundstruktur der \texttt{SimulatorInstance} beschrieben, welche Komponenten diese verwenden kann und wie letztendlich die unterschiedlichen Instanzen aufgeteilt sind. 

Damit jede Simulatorinstanz von der \texttt{SimulatorMain} verwendet werden kann, implementieren diese ein gemeinsames Interface. Dieses heißt \texttt{ISimulator} und erweitert das Interface \texttt{ISimulatorViewAPI}. Die \texttt{ISimulatorViewAPI} definiert die Schnittstelle zwischen GUI-Komponente und der \texttt{SimulatorInstance}. Zum einen muss auf das Modell zugegriffen werden. Dieses wird über die Methode \texttt{getModel()} bereitgestellt. Des Weiteren wird noch ein ITargetUpdateListener benötigt, der vom ScenarioUpdater verwendet wird um Ziele über das Asterix-Interface zu versenden. Dieser wird mit der Methode \texttt{getTargetUpdater()} übergeben. Diese beiden Methoden bilden die \texttt{ISimulatorViewAPI}.

Der AbstractSimulatorMain benötigt den ISimulator, um diesen zu stoppen und um die GUI-Komponente die ISimulatorViewAPI übergeben. Deswegen ist die einzige Methode des Interfaces die stop()-Methode. Das Starten der SimulatorInstanz wird vom RestartService übernommen und wird in Kapitel 4.X gezeigt.

TODO

\subsection{Simulator Core}
Eine Komponente, die jede Instance enthält, ist das Datenmodell. Es wurde entschieden, dass das Modell des RadarSimulator mit einer kleinen Erweiterung, für alle drei Anwendungen genutzt wird. Dadurch kann nicht nur der Modell-Code wiederverwendet werden, sondern auch der Code zum Aufbauen und Bearbeiten des Modells bleibt identisch.

Da das Modell in jeder Anwendungsausprägung verwendet, ist es sinnvoll eine Klasse zu erstellen, die das Modell verwaltet und auf der die weiteren Simulator Instanzen aufbauen. Diese Klasse heißt \texttt{SimulatorCore}. Der \texttt{SimulatorCore} besitzt eine Methode zu Erstellen des Modells, welche nur von den erbenden Klassen aufgerufen werden kann. Auf Abbildung \ref{fig:SimulatorInstance} ist zu sehen, wie diese das \texttt{ISimulator-Interface} implementiert.


\begin{figure}[p]
    \centering
    \includegraphics[width=0.85\textwidth]{content/assets/Kapitel4/SimulatorInstance.png}
    \caption{Die ausprägungen SimulatorInstance}
    \label{fig:SimulatorInstance}
\end{figure}

\subsection{Modell-Observer im SimulatorCore}
Eine weitere Funktion, welche im \texttt{SimulatorCore} implementiert wird, ist die des Modell-Observers. Wie in Kapitel 3.X gezeigt wurde, hat die Benutzeroberfläche des RadarSimulator zwei Abhängigkeiten. Das Model und den Sensor. In den neuen Anwendungen ist die GUI-Komponente nur noch vom Modell abhängig. 

\begin{figure}[ht]
    \centering
    \includegraphics[width=0.7\textwidth]{content/assets/Kapitel4/AlterSimulator.png}
    \caption{Reaktion des alten RadarSimulator auf GUI-Änderung}
    \label{fig:SimulatorAlt}
\end{figure}

Um die Abhängigkeit der Benutzeroberfläche vom Sensor zu lösen, wird das Modell von so genannten Observern überwacht. Die Observer beobachten ein festgelegtes Objekt des Modells und führen bei einer Änderung dieses Objekts eine Methode aus. In diesen Methoden soll entweder der Sensor gesteuert oder die neue Netzwerkschnittstelle bedient werden. Diese Observerfunktion befindet sich ebenfalls im \texttt{SimulatorCore}. Im Sequenzdiagramm \ref{fig:SimulatorAlt} sieht man die Funktionsweise des alten Simulators. Im Sequenzdiagramm \ref{fig:SimulatorNew} die des neuen Simulators.

\begin{figure}[ht]
    \centering
    \includegraphics[width=0.85\textwidth]{content/assets/Kapitel4/NeuerTestSimulator.png}
    \caption{Reaktion der neuen Simulatoren auf GUI-Änderung}
    \label{fig:SimulatorNew}
\end{figure}


Zur besseren Übersicht des Codes, gibt es für verschiedene Bereiche des Modells eigene Observer. So gibt es z.B. für die Bitfehler einen \texttt
{BitModelObserver}. Jeder dieser Observer bekommt ein Interface übergeben, welches die verschiedenen Reaktionen des Observers, implementieren soll. Dieses Interface ist in Abbildung \ref{fig:SimulatorNew} als ModelChangeHandler zu erkennen. Um das an einem Beispiel zu verdeutlichen, wird der \texttt{BitModelObserver} des TrainerSimulator betrachtet. 

\begin{figure}[ht]
    \centering
    \includegraphics[width=1\textwidth]{content/assets/Kapitel4/BitChangeBeispiel.png}
    \caption{Reaktion des TrainerSimulator auf Bitänderung}
    \label{fig:TrainerNewBit}
\end{figure}

\sloppy
Der \texttt{BitModelObserver} beobachtet die \texttt{DefectReports} Liste im Modell. Mögliche Änderungen, die den Observer notifizieren sind das Hinzufügen, Ändern oder das Entfernen eines BitDefects. Damit auf diese Änderungen reagiert werden kann, wird ein Interface zur Verfügung gestellt. Dieses Interface heißt in diesem Fall \texttt{IBitModelChangeHandler}.
Die spezifischen Instancen müssen ein Objekt der Klasse \texttt{IBitModelChangeHandler} erstellen und übergeben. Dieses wird über die abstrakte Methode \texttt{getBitChangeHandler()} des \texttt{SimulatorCore} übergeben, welche jeder der erbenden Klassen implementieren muss. Der TrainerSimulator übergibt ein Objekt der Klasse \texttt{TrainerBitChangeHandler}, welches Nachrichten über den Server versendet.

\subsection{TrainerSimulator}
Wie man auf Abbildung \ref{fig:SimulatorInstance} erkennen kann erbt sie Instance-Klasse des Trainersimulators direkt vom \texttt{SimulatorCore} und heißt \texttt{TrainerSimulator}. Dieser erweitert den \texttt{SimulatorCore} um einen \texttt{TrainerServer}. Dieser ist dafür zuständig, dass veränderte Daten im Modell an die Trainees versendet werden.

Jeder der verwendbaren SimulatorInstancen besitzt eine Methode namens \texttt{start}, um die Instanz zu starten. Als Parameter der start-Methode werden Verbindungsdaten und Sensorinformationen eingeben. Beim Aufruf der \texttt{start()} Methode wird zu Beginn das Modell, abhängig von den Sensorinformationen erstellt und gefüllt. Daraufhin wird ein \texttt{ITrainerServer}-Objekt erzeugt und gestartet. Nachdem Modell und Server initialisiert sind, werden die Modellobserver erstellt, welche die \texttt{TrainerModelChangeListener} übergeben bekommen. Die \texttt{TrainerModelChangeListener} verwenden den \texttt{TrainerServer}, um Nachrichten an die Clients zu senden. Neben dem Observer wird noch ein lokaler Ordner erstellt, um Szenarien, die der Trainer im Benutzerinterface erstellt hat, zu speichern.

Das\texttt{ISimulatorViewAPI}-Interface des \texttt{TrainerSimulators} benötigt noch eine  implementierte {getTargetUpdater()}-Methode, welche dafür zuständig ist, dass TargetSimulations, die vom \texttt{ScenarioUpdater} kommen, versendet werden können. Auf Abbildung \ref{fig:TrainerTargetSimulation} ist zu erkennen wie der \texttt{ScenarioUpdater}, dem ihm übergebenen \texttt{ITargetUpdaterListener} aktiviert. In diesem Fall ist das der \texttt{TargetSimulationNetworkUpdater}, welcher Nachrichten über den \texttt{ITrainerServer} sendet.

\begin{figure}[ht]
    \centering
    \includegraphics[width=0.9\textwidth]{content/assets/Kapitel4/TargetSimulationNetwork.png}
    \caption{TrainerSimulator sendet TargetSimulation}
    \label{fig:TrainerTargetSimulation}
\end{figure}

\subsection{AsterixSimulator}
Der Trainee-, sowie der Testsimulator bauen eine Verbindung zur Venus auf. Deshalb benötigen deren Instanzen einene \texttt{AbstractSensor}, welcher ein Asterix-Interface zur Verfügung stellt. Außerdem verwenden beide Anwendungen einen \texttt{TrackUpdater}, einen \texttt{BeamlineTask} und einen \texttt{moxaAndCameraServerTask}, die den \texttt{AbstractSensor} steuern un das Verhalten eines realen Sensors simulieren. Neben dem \texttt{AbstractSensor} brauchen beide auch einen entsprechenden \texttt{CommandListener}, der auf eingehende Kommandos des Asterix-Interfaces reagiert und passende Modelländerungen vornimmt.

Damit diese Funktionen nicht für die \texttt{TraineeInstance} und die \texttt{TestInstance} einzeln definiert werden müssen, wird die Komponente \texttt{AsterixSimulator} erstellt. Diese übernimmt die aufgezählten Funktionen und erbt von der Klasse \texttt{SimulatorCore}. Somit enthält diese auch das Model, Observer und den Code zur Unterstützung.

Wie der \texttt{TrainerSimulator}, muss auch ein \texttt{AsterixSimulator} auf Modelländerungen reagieren. Modeländerungen des Test- und des \texttt{TraineeSimulators} werden gleich gehandhabt. Auf eine Modelländerung folgt eine Asterix-Nachricht. Dabei gibt es ein paar Ausnahmen, bei denen keine Nachricht versendet werden muss. Diese werden durch den Modell-Observer geregelt, indem dieser nur relevante Modelldaten observiert. Die Objekt, welches die Modelländerungen handhaben, müssen das \texttt{IModelChangeListener}-Interface implementieren und werden über die abstrakten Klassen \texttt{getObserverChangeListener()} übergeben. Im Sequenzdiagramm \ref{fig:AsterixSimulator} wird dargestellt, wie der Ablauf nach einer Modelländerung ist. 

\begin{figure}[ht]
    \centering
    \includegraphics[width=0.9\textwidth]{content/assets/Kapitel4/AsterixSimulator.png}
    \caption{AsterixSimulator reagiert auf Modelländerung}
    \label{fig:AsterixSimulator}
\end{figure}

\subsection{TraineeSimulator}

Der \texttt{TraineeSimulator} erweitert den \texttt{AsterixSimulator}. Diese Instanz muss eine Verbindung zum \texttt{TrainerServer} aufbauen. Dafür verwendet der \texttt{TraineeSimulator} einen \texttt{TraineeClient}. Erst nachdem dies geschehen ist kann das Modell des Trainees geladen werden, da dieser den Sensortyp und die AsterixVersion vom Trainer übertragen bekommt.

Wenn der Server Daten an die verbundenen \texttt{TraineeClients} versendet, wird deren \texttt{MessageReceivedHandler} ausgelöst. Dies passiert über das Interface \texttt{IMessageRecievedListener}, welches die Methode \texttt{messageRecieved()} implementiert. Der \texttt{MessageReceivedHandler} entscheidet, wie mit den empfangenen Nachrichten umgegangen wird. Es gibt drei Arten von Nachrichten. Diese sind:

\begin{enumerate}
    \item Nachrichten, die \texttt{Sensorinformationen} beinhalten.
    \item Nachrichten, die das Modell updaten.
    \item Nachrichten, die \texttt{TargetSimulations} für den \texttt{TrackUpdater} beinhalten
\end{enumerate}

Nachrichten der 1. Art werden empfangen, wenn eine Verbindung zu einem Server aufgebaut wird. Daraufhin wird ein neues Modell und ein neuer \texttt{AbstractSensor} basierend auf den Informationen erstellt. Die Nachrichten 2. Art werden einfach dem Modell hinzugefügt und die der 3. Art dem \texttt{TrackUpdater}.

\subsection{TestSimulator}
Die Instance des \texttt{TestSimulators} baut auch auf dem \texttt{AsterixSimulator} auf. Die Klasse benötigt lediglich eine \texttt{start()}- Methode und muss einen Ordner erstellen, in dem Scenarios gespeichert werden. Dadurch funktioniert der \texttt{TestSimulator} wie der \texttt{RadarSimulator}.
\newpage
\section{Netzwerkschnittstelle}
Die Netzwerkschnittstelle soll die Übertragung der Daten vom Trainer zum Trainee ermöglichen. Dafür wird ein eigenes Plug-In angelegt, welches die Funktionalität der Netzwerkschnittstelle zur Verfügung stellt. Indem man die Funktionalität in ein Plug-In auslagert, ist es möglich Code hinter einem Interface zu verbergen. Dadurch ist dieser für andere Plug-Ins nicht sichtbar und die Funktionalität wird nur über eine spezifische Schnittstelle bereitgestellt. Daraus resultiert eine modulare Softwarekomponente.
Auf der Abbildung \ref{fig:NetworkInterface} hat man eine Übersicht des Moduls und über dessen Packages. Die Packages, die den Rand des Moduls berühren sind für weitere Module verwendbar.

\begin{figure}[ht]
    \centering
    \includegraphics[width=0.9\textwidth]{content/assets/Kapitel4/NetworkInterface.png}
    \caption{Struktur des Netzwerk-Interface Moduls}
    \label{fig:NetworkInterface}
\end{figure}

\subsection{Factories}
Die Objekte, die vom \texttt{TrainingSimulatorNetworkInterface}-Modul geliefert werden sollen, sind ein \texttt{TrainerServer} vom Typ \texttt{ITrainerServer} und ein \texttt{TraineeClient} vom Typ \texttt{ITraineeClient}. Damit ein \texttt{ITrainerServer} oder ein \texttt{ITraineeClient}, einem anderen Plug-In zur Verfügung gestellt werden können, werden Factory-Klassen verwendet. Die Funktion der Factory-Klasse wird bam Beispiel der \texttt{TrainerServerFactory} erklärt.

Außerhalb des TrainingSimulatorNetworkInterface-Moduls soll die Klasse TrainerServer nicht verwendet werden. Um ein Objekt vom ITrainerServer zu erstellen benötigt man jedoch einen Konstruktor einer Klasse, die das Interface implementiert. In diesem Fall ist das die TrainerServer-Klasse. Damit ein TrainerServer-Objekt erstellt werden kann, ohne die Klasse zu verwenden, benötigt man eine Factory. Diese befindet sich im selben Modul und hat somit Zugriff auf die Klasse. Die Factory enthält die Methode getInstance(IChannelActivatedListener), die ein Objekt vom Typ ITrainerServer zurückliefert. In der Methode wird der Konstruktor des TrainerServer aufgerufen und es wird ein Objekt erstellt, welches zurückgegeben wird. Damit die Methode ausgeführt werden kann wird noch ein IChannelActivatedListener-Objekt benötigt. Dieses muss von dem Modul zur Verfügung gestellt werden, welche den ITrainerServer erzeugen will. Das Objekt stellt eine Methode bereit, welche aufgerufen wird, wenn der Server eine Verbindung zu einem Client aufgebaut hat.

Die Module, die das TrainingSimulatorNetworkInterface-Modul verwenden, kennen das Interface ITrainerServer sowie die TrainerServerFactory-Klasse, da diese vom Plug-In exportiert werden. Somit kann ein TrainerServer in einem anderen Modul über das Interface ITrainerServer verwenden, ohne dass dieses Modul die Klasse TrainerServer kennt. 

Die Factory des Clients funktioniert nach demselben Prinzip, bis auf das anstatt eines IChannelActivatedListener-Objekts, ein IMessageReceivedListener-Objekt übergeben wird. Dieses implementiert eine Methode, welche aufgerufen wird, wenn eine Nachricht empfangen wurde.
\newpage
\section{Features}

\subsection{Single Target Tracking}
Um Single Target Tracking zu ermöglichen, benötigt die Anwendung eine Erweiterung um zwei Funktionen. Zum einen muss überprüft werden, ob sich ein Ziel 
im Suchbereich befindet, wenn dies zutrifft, soll nach jeder Bewegung des Ziels, das Audiogate auf dessen Position gesetzt werden. Des Weiteren muss sich 
die Beamline, je nach Zielverfolgung oder Zielsuche korrekt verhalten.

Während der STT Modus aktiviert ist, bewegt sich die Beamline innerhalb eines engen Sektors wiederholt über das Ziel. Wenn kein Ziel gefunden wird, sucht
die Beamline in einem größeren Bereich nach einem neuen Ziel. Der Sektor hat das Audiogate als Mittelpunkt und ändert seinen Bereich synchron zum 
Audiogate. Wenn ein Ziel verfolgt wird beträgt die Breite des Sektors 100 mils. Falls der Sensor ein Ziel zur Verfolgung sucht, beträgt die 
Bewegungsweite 300 mils. Diese Funktion wird im BeamlineTask implementiert, da dieser Zugriff auf das Audiogate und die Beamlineinformationen hat. Der 
BeamlineTask hat jedoch keinen Zugriff auf die Daten der Tracks. Deshalb weiß er nicht, ob ein Ziel verfolgt wird und kann kein Audiogate setzen. Damit 
der Beamlinetask weiß, ob ein Ziel verfolgt wird, wird ein Booleanflag, welches jederzeit abgefragt werden kann, im Model hinzugefügt.

Das Flag, sowie das Audiogate sollen vom Trackupdater gestetzt werden. Dieser hat alle Informationen der verfügbaren Tracks, die in der Simulation 
existieren und kann berechnen, ob das Ziel derzeitig von der Beamline erfasst wird. Der TrackUpdater hat die Methode schedule(), welche wiederholt nach 
einem bestimmten Zeitintervall aufgerufen wird. Diese Methode aktualisiert die Tracks und sendet Track Informationen an den Sensor. Diese Methode wird 
nun erweitert, um das STTFlag und das Audiogate zu setzen. Ein verfolgter Track wird im TrackUpdater gespeichert.

Zum Anfang der Methode wird überprüft, ob bereits ein Track verfolgt wird, falls dies nicht der Fall ist wird die Distanz vom Audiogate zu allen 
vorhandenen Tracks berechnet. Die Werte werden in einer Liste mit dem jeweiligen Track gespeichert, wenn sie innerhalb des STT-Sektors liegen. Die 
STT-Funktion ist so definiert, dass sie den nächsten Track zum Audiogate innerhalb des Sektors so lange verfolgt, bis dieser verschwindet oder man das 
Audiogate manuell ändert. Deshalb wird die Liste so sortiert, damit der Track mit dem geringsten Abstand zum Audiogate an erster Stelle steht. Nachdem 
sortiert wurde, wird das erste Element gespeichert, das Flag wird auf true gesetzt und die Liste wird geleert. Wenn die Liste leer ist wird nichts getan. 

Wird bereits ein Track verfolgt, wird geprüft, ob der zu verfolgende Track aktuell existiert. Das passiert indem man in der Liste, der gegenwärtigen 
Tracks nach einem Track mit der identischen ID des gespeicherten Tracks ist sucht. Ist dieser vorhanden, wird dieser Track als neuer Track gespeichert. 
Als nächstes wird getestet, ob dieser Track vom Sensor detektiert wird.  Trifft es zu, dass der Track vorhanden ist und detektiert wird setzt der 
TrackUpdater das Audiogate auf die aktuelle Position des Tracks. Wenn der Track nicht detektiert wird oder existiert, wird ein Zähler hochgezählt. Es 
kann passieren, dass ein Track in einem oder mehreren Durchläufen nicht im Bereich der Beamline liegt oder dass der Track vom Sensor nicht erfasst wird, 
da er z.B. zu weit entfernt ist. Durch den Zähler kann man einen Toleranzwert festlegen, wie oft der Track nicht erkannt werden muss bevor diesem nicht 
mehr gefolgt wird. Dem Track wird entfolgt, indem das Flag auf false gesetzt und der Track gelöscht wird.

\begin{figure}[ht]
    \centering
    \includegraphics[width=0.6\textwidth]{content/assets/BeamlineTaskSTT.png}
    \caption{Flussdiagramm des Algorithmus des BeamlineTask}
\end{figure}

\begin{figure}[ht]
    \centering
    \includegraphics[width=0.65\textwidth]{content/assets/TrackUpdaterSTT.png}
    \caption{Flussdiagramm des Algorithmus des Trackupdaters}
\end{figure}




