\subsection{SimulatorInstance}
Die SimulatorInstance unterscheidet sich bei jeder Anwendungsausprägung. Jedoch gibt es Komponenten, welche von mehreren der Anwendungen verwendet werden 
können. Im Folgenden wird die Grundstruktur SimulatorInstance beschrieben, welche Komponenten diese verwenden kann und wie letztendlich die 
unterschiedlichen Instancen aufgeteilt sind.

Eine Komponente, die jede Instance enthält, ist ein Datenmodell. Es wurde entschieden, dass das Modell des RadarSimulators mit einer kleinen Erweiterung,
 für alle drei Anwendungen genutzt wird. Dadurch kann nicht nur der Modell-Code wiederverwendet werden, sondern auch der Code zum Erstellen und Verwalten 
des Modells bleibt identisch.

Da es nun eine Komponente gibt, die jede Anwendungsausprägung verwendet, ist es sinnvoll eine Klasse zu erstellen, die das Modell verwaltet und auf der 
die weiteren Simulator Instanzen aufbauen. Diese Klasse heißt SimulatorCore.

