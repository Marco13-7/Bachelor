\subsection{SimulatorInstance}

Die \texttt{SimulatorInstance} übernimmt die gesamte Simulation des Sensors der Anwendung und befindet sich, wie auf Abbildung \ref{img:SimulatorModels} im Plug-In \texttt{SensorSimulation}. Sie hat folgende Aufgaben:

\begin{itemize}
    \item Erstellen des Datenmodells des Simulators. Aufgrund von unterschiedlichen Werten und Zubehör zwischen den Sensortypen, wird das       Datenmodell je nach Sensortyp und Asterix-version entsprechend befüllt. Dafür wird eine Helferklasse verwendet.
    \item Erstellen eines Sensors, je nach Sensortyp wird ein passender Sensor initialisiert. 
        Der \texttt{AbstractSensor} ist die Vaterklasse der jeweiligen Sensortypen (\texttt{Sensor BORA}, \texttt{SensorGO12}, \texttt{SensorGO80} und \texttt{SensorGA10}). Über den Sensor können Asterix-Kommandos an die Venus gesendet werden
    
    \item Erstellen und starten des \texttt{BeamlineTask}. Der \texttt{BeamlineTask} ist dafür zuständig, dass sich die Beamline des Sensors korrekt bewegt und die dazugehörigen Werte im Modell geupdatet werden.
    \item Erstellen und starten des \texttt{TrackUpdater}. Er ist dafür zuständig die Tracks korrekt upgedatet werden und über das Asterixinterface versendet werden.
    \item Erstellen und starten des \texttt{MoaxaAndCameraServerTask}. Dieser simuliert optionales Zubehör und die Kamera, falls dieses vorhanden ist.    
    
\end{itemize}

Das Datenmodell, sowie die Tasks enthalten wichtige Funktionen, die von der Test- und Traineeanwendung verwendet werden. Deshalb ist es sinnvoll diese zu übernehmen und falls nötig umzuändern.
