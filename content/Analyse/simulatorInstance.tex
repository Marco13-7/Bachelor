\subsection{SimulatorInstance}
Die SimulatorInstance hat folgende Aufgaben:
\begin{itemize}
    \item Erstellen des Datenmodells des Simulators. Aufgrund von unterschiedlichen Werten und Zubehör zwischen den Sensortypen, wird das Datenmodell je nach Sensortyp und Asterix-version entsprechend befüllt. Dafür wird eine Helferklasse verwendet.
    \item Erstellen eines Sensors, je nach Sensortyp wird ein passender Sensor initialisiert. 
    Der AbstractSensor ist die Vaterklasse der jeweiligen Sensortypen (\texttt{SensorBORA}, SensorGO12, SensorGO80 und SensorGA10). Über den Sensor können \\Asterix-Kommandos an die Venus versendet werden
    \item Erstellen und starten des BeamlineTasks. Der BeamlineTask ist dafür zuständig, dass sich die Beamline des Sensors korrekt bewegt und die entsprechenden Werte im Modell geupdatet werden.
    \item Erstellen und starten des Trackupdater. Er ist dafür zuständig die Tracks korrekt upgedatet werden und über das Asterixinterface versendet werden.
    \item Erstellen und starten des MoaxaAndCameraServerTask. Dieser simuliert optionales Zubehör und die Kamera, falls dieser vorhanden ist.    
    
\end{itemize}

Diese Tasks haben eine klar definierte Aufgabe und lassen sich gut wiederverwenden.