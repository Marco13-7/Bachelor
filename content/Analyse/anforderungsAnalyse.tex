\section{Anforderungsanalyse der Ausprägungen/Anwendungen}
\subsection{Funktionale Anforderungen}

In diesem Abschnitt werden die funktionalen Anforderungen der drei verschiedenen Anwendungen, die im Rahmen dieser Arbeit entstanden sind, vorgestellt.
„Funktionale Anforderungen beschreiben gewünschte Funktionalitäten (was soll das System tun/können) eines Systems bzw. Produkts, dessen Daten oder 
Verhalten.“

Die funktionalen Anforderungen des Trainersimulators sind:

\begin{itemize}
    \item Der Trainersimulator muss dem Trainee ermöglichen sich mit dem Trainersimulator zu verbinden, wenn dieser sich im selben Netzwerk befindet und
     die korrekte IP-Adresse eingegeben wurde.
    \item Der Trainersimulator muss, bei Änderungen von PNU-Daten, dem hinzufügen von Bitfehlern oder dem Abspielen von Szenarien durch die graphische 
    Oberfläche, die entsprechenden Informationen über das Netzwerk an alle verbundenen Trainees versenden.
\end{itemize}

Der Traineesimulator hat folgende funktionale Anforderungen:

\begin{itemize}
    \item Der Traineesimulator soll sich mit dem Trainersimulator verbinden können.
    \item Der Trainingssimulator soll sich mit der Venus verbinden, nachdem dieser die nötigen Informationen vom Trainersimulator empfangen hat.
    \item Der Trainingssimulator muss, wenn er Simulations-Daten empfängt, diese verarbeiten und über das Asterix-Interface versenden
    \item Der Trainingssimulator muss, wenn er Model-Daten empfängt, die entsprechenden Informationen im Modell updaten und über das Asterix-Interface
     senden
\end{itemize}

Die funktionalen Anforderungen der Testanwendung sollen nicht von den funktionalen Anforderungen des Radar Simulators nicht abweichen. Diese lauten:

\begin{itemize}
    \item Der Testsimulator soll eine Verbindung mit der Venus über das \\ Asterix-Interface aufbauen.
    \item Der Testsimulator muss Informationen, die auf der graphischen Benutzeroberfläche verändert wurden über das Asterix-Interface versenden.
\end{itemize}

\subsection{Nicht funktionale Anforderungen}

„Nichtfunktionale Anforderungen sind Anforderungen, an die "Qualität" in welcher die geforderte Funktionalität zu erbringen ist“. Die nicht funktionalen 
Anforderungen lassen sich für die drei Applikationen zusammenfassen, da diese fast vollständig übereinstimmen. Lediglich durch die Netzwerkverbindung der 
Trainer- und Traineeanwendung entstehen Unterschiede. Die nichtfunktionalen Anforderungen lassen sich in mehrere Bereiche unterteilen. 
Leistungsanforderungen, Qualitätsanforderungen und Randbedingungen.

„Unter Leistungsanforderungen versteht man im Allgemeinen, Anforderungen an die empirisch messbaren nicht-funktionalen Anforderungen eines Systems, also 
Anforderungen, deren zu Grunde liegendes Bedürfnis ein Leistungsmerkmal ist.“ [Braun-ausarbeitung]. Eine Leistungsanforderung jeder Anwendung ist, dass 
diese die Simulation in Echtzeit ausführt und eine Verzögerung um 10 ms im akzeptablen Bereich liegt. Zudem sollen diese nicht zu viel Speicher besetzen 
und die Prozessorbelastung sollte einen realistischen Wert betragen. Um dafür zu garantieren werden Belastungstests, mit der entsprechenden Hardware, auf 
der die Anwendung später laufen soll, durchgeführt.

Der „ISO/IEC 25000 Software engeneering Software product Qualitiy Requirements and Evaluation (SQUuarRE) ist ein aktueller Standard für die 
Qualitätskriterien und -bewertungen von Softwareprodukten. Der Software Product Quality Model aus dem Standard ISO/IEC 25010 beschreibt acht Kriterien, 
um die Qualität eines Produktes zu bewerten. Die acht Kriterien sind:
\begin{itemize}
    \item Funktionalität
    \item Leistungseffizienz
    \item Kompatibilität
    \item Benutzbarkeit
    \item Zuverlässigkeit
    \item Sicherheit
    \item Wartbarkeit
    \item Übertragbarkeit    
\end{itemize}

Die Qualitätsanforderung, die in dieser Arbeit im Mittelpunkt steht, ist die Wartbarkeit der Anwendungen. Wie bereits erwähnt soll die Arbeit zeigen, wie 
eine Anwendung modularisiert und erweitert werden kann. Diese Kriterien fallen unter den Oberbegriff Wartbarkeit. Vor allem die Wiederverwendbarkeit von 
Komponenten ist ein wichtiges Kriterium beim Erstellen der Anwendungen.
