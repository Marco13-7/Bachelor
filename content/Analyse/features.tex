\chapter{Analyse der Features}
\subchapter{Single Target Tracking}
Ein Ziel in der Venus wird ausgewählt und der STT Modus wird aktiviert. Das Audiogate wird auf das Ziel gesetzt. Es wird ein extra Sektor mit einem
Winkel von 100 mils erstellt, welcher das Audiogate als Mittelpunkt hat. Die Beamline bewegt sich innerhalb des Sektors und erfasst das Ziel erneut. Wenn
das Ziel wieder erfasst wird, wird das Audiogate erneut auf dem Ziel positioniert. Falls eine gewisse Zeit kein \texttt{Track} gefunden wird soll das Radargerät
den Sektor vergrößern und einen neuen Track suchen.

\subchapter{Doppler Ton}
Durch den DopplerEffekt lässt sich erkenn, ob sich ein Objekt bewegt oder still steht. Bevor die Technologie soweit entwickelt war, dass man Objekte auf
graphischen Oberflächen anzeigte, wurden Radarsignale in Audiosignale umgewandelt. Durch diese Signale konnten gut trainierte Soldaten die vom Radar 
erfassten Objekte identifizieren. Je nach Oberfläche, Größe und die Bewegungen unterscheiden sich das Radarsignal, sowie das Audiosignal. So konnte z.B.
ein Auto von einem Hubschrauber unterschieden werden. Dieses Audiosignal wird auch von den Sensoren von Thales bereitgestellt. Ebenso gibt es auch in der 
Venus eine Möglichkeit dieses Signal abspielen zu lassen, wenn ein Ziel mit Hilfe von Single Target Tracking verfolgt wird. Diese Funktion ist ebenfalls
nicht im Simulator implementiert. 
