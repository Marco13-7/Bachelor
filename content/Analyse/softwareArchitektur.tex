\section{Analyse des Radarsimulators}

\subsection{Softwarearchitektur}

Im folgenden Abschnitt werden die einzelnen Softwarekomponenten der Radarsimulator Anwendung beschrieben und auf Wiederverwendbarkeit und Modularität überprüft. Dabei stehen die Kernkomponenten der Anwendung im Fokus. Der Radarsimulator wird von Thales zur Verfügung gestellt und wurde vor und während dem Schreiben des Kapitels nicht verändert. Der Radarsimulator mit neuer Architektur wird in den folgenden Kapiteln Testsimulator genannt.

Der Radarsimulator ist eine Eclipse-RCP Anwendung und setzt sich aus mehreren Equinox Plug-In-Modulen zusammen. Er wird durch Basis-Plug-Ins der Venus unterstütz. Diese stellen spezifische Funktionen, wie Anwendungsstruktur und Helferklassen zur Verfügung. Der Startpunkt der Anwendung ist die Klasse \texttt{SimulatorMain}, welche sich in dem Modul \texttt{RadarSimulator.app} befindet. Dieses stellt die Benutzeroberfläche des Simulators zur Verfügung. Im Modul werden die Startklasse der Anwendung und weitere Fenstereinstellungen für diese Anwendungen in einer \texttt{Application.e4xmi-Datei} angegeben. Diese Datei ist die Basis jeder Eclipse-RCP Applikation. Wenn die Eclipse Applikation die Startklasse in Java aufruft, startet der RestartService. Diese ist im folgenden Codeabschnitt zu sehen.


\begin{lstlisting}

@PostConstruct
void start(final BorderPane parent) {
    this.parent = parent;
    restartService.setOnScheduled(e -> {
        parent.setCenter(new JFXSpinner());
        disposeOverview();
    });
    restartService.setOnSucceeded(e -> {
        instance = restartService.getValue();
        initSimulatorOverview(instance);
    });
    restartService.restart();
}

\end{lstlisting}

Die gesamte Funktionalität der Sensorsimulation stellt die \texttt{SimulatorInstance zur Verfügung}. Wenn der \texttt{RestartService} die \texttt{SimulatorInstance} erfolgreich geladen hat, initialisiert dieser die Benutzeroberfläche der Anwendung. Außerdem ermöglicht es der \texttt{RestartService}, die Simulation durch einen Mausklick in ein Fenster Menü item neu zu starten. Dabei werden die Benutzeroberfläche, sowie die \texttt{SimulatorInstance} neu geladen. Der \texttt{RestartService} soll zudem auch im zukünftigen Testsimulator sowie im Trainer- und Traineesimulator enthalten sein. Die \texttt{SimulatorInstance} und der \texttt{SimulatorMainContentProvider} werden in der SimulatorMain als Attribute gespeichert. Wenn die Benutzeroberfläche geladen wird, bekommt diese die \texttt{SimulatorInstance} übergeben, damit diese die Simulation steuern kann. Wie es bei einer Eclipse Applikation üblich ist, sind Funktionen der Anwendung in Softwaremodule eingeteilt. Diese Module heißen im Equinox-Framework Plug-Ins. Auf Abbildung \ref{img:SimulatorModels} erkennt man die Aufteilung der Hauptkomponenten in zwei Plug-Ins.

\begin{figure}[h]
    \centering
    \includegraphics[width=0.8\textwidth]{content/assets/Kapitel3/SimulatorModels.png}
    \caption{Die Module \texttt{SensorSimulator} und \texttt{RadarSimulator.app} des Radarsimulators}
    \label{img:SimulatorModels}
\end{figure}


\subsection{Benutzeroberfläche}
Das Frontend des RadarSimulators verwendet das Framework JavaFX und wird über den SimulatorMainContentProvider bereitgestellt. Der SimulatorMainContentProvider enthält verschiedene Einstellungsmöglichkeiten und stellt Klassen, die die Funktionalität des   Benutzerinterface implementieren zur Verfügung. Diese Klassen heißen FXML Controller. Jeder FXML Controller ist mit einer FXML-Datei verknüpft, die die Anordnung der FXML Komponenten definiert. Der oberste FXML Controller des Radarsimulators ist der SimulatorMainController. 

Die Benutzeroberfläche besteht aus mehreren einzelnen Tabs, in denen die Funktionen des Simulators in unterschiedliche Bereiche eingeteilt werden. Jeder Tab hat einen Haupt FXML Controller. Wenn der Controller einen zu großen Funktionsumfang hat, kann Funktionalität auf weitere FXML Controller aufgeteilt werden. 

Da der SimulatorMainController der oberster FXML Controller ist, werden in ihm alle möglichen Tabs des Simulators vordefiniert und je nach Bedarf initialisiert. Welche Tabs geladen werden hängt davon ab, welcher Sensortyp und welche Asterixversion ausgewählt ist. Für jeden einzelnen Tab gibt es einen weiteren JavaFx Controller, in dem die Funktionalität der Oberfläche implementiert ist. Zwischen Backend und Frontend gibt es keine definierte Schnittstelle. Die Simulatorinstanz wird beim Erzeugen des Controllers an das Frontend übergeben. Das heißt die FXML Controller haben immer vollen Zugriff auf die Simulatorinstanz. Das ist nicht optimal, da Abhängigkeiten schnell unübersichtlich werden können und die Wartbarkeit des Systems erhöhen.

Um das genauer zu erläutern wird das Beipiel aus dem Sequenzdiagramm ABBL. Bit genauer betrachtet. Auf der GUI-Oberfläche wird nun ein Bitfehler hinzugefügt. Daraufhin reagiert der entsprechende actionListener. Er fügt den Bitfehler dem Model hinzu und sendet diesen über das Asterix-Interface an die Venus. Wenn die GUI Komponente das Model und den Sensor kontrolliert hat diese keine klare Aufgabe definiert. Bei einer Modularen Architektur ist es wichtig, dass jede Komponente „eine klar abgegrenzte Aufgabe“[Quelle 2] erfüllt. Somit ist diese Architektur fehlerhaft. Inwiefern die Funktion der View-Komponente verändert werden muss damit diese die Anforderungen einer modularen Architektur erfüllt wird in Kapitel 4.X geschildert.

\begin{figure}[h]
    \centering
    \includegraphics[width=0.8\textwidth]{content/assets/Kapitel3/AlterSimulator.png}
    \caption{UI - Alter Simulator}
\end{figure}

Weitere Optimierung bedarf es bei der Initialisierung der Tabs. Die späteren Anwendungen haben unterschiedlich viele Tabs. Die genaue Übersicht gibt es 
in ABBL. X

\begin{figure}[h]
    \centering
    \includegraphics[width=0.8\textwidth]{content/assets/Kapitel3/TabsPerApp.png}
    \caption{UI - Alter Simulator}
\end{figure}

Deshalb ist es entweder notwendig für jede Anwendung einen separaten SimulatorMainController zu konfigurieren. Daraus würde redundanter Code entstehen, 
den man so gut es geht, vermeiden möchte. Die Alternative wäre eine allgemeinen SimulatorMainController, welcher alle verfügbaren Tabs im OSGi-Kontext 
lädt. Um das zu implementieren können Extension Points verwendet werden.

\subsection{SimulatorInstance}
Die SimulatorInstance hat folgende Aufgaben:
\begin{itemize}
    \item Erstellen des Datenmodells des Simulators. Aufgrund von unterschiedlichen Werten und Zubehör zwischen den Sensortypen, wird das Datenmodell je nach Sensortyp und Asterix-version entsprechend befüllt. Dafür wird eine Helferklasse verwendet.
    \item Erstellen eines Sensors, je nach Sensortyp wird ein passender Sensor initialisiert. 
    Der AbstractSensor ist die Vaterklasse der jeweiligen Sensortypen (\texttt{SensorBORA}, SensorGO12, SensorGO80 und SensorGA10). Über den Sensor können \\Asterix-Kommandos an die Venus versendet werden
    \item Erstellen und starten des BeamlineTasks. Der BeamlineTask ist dafür zuständig, dass sich die Beamline des Sensors korrekt bewegt und die entsprechenden Werte im Modell geupdatet werden.
    \item Erstellen und starten des Trackupdater. Er ist dafür zuständig die Tracks korrekt upgedatet werden und über das Asterixinterface versendet werden.
    \item Erstellen und starten des MoaxaAndCameraServerTask. Dieser simuliert optionales Zubehör und die Kamera, falls dieser vorhanden ist.    
    
\end{itemize}

Diese Tasks haben eine klar definierte Aufgabe und lassen sich gut wiederverwenden.
\subsection{Model}


Das Datenmodell des Radarsimulators wird mithilfe des Frameworks EMF erstellt. Das Modell befindet sich in einem eigenen Plug-In Modul. Darin sind ein graphisches Entity-Relation-Diagramm des Modells und der daraus generierbare Modellcode enthalten. Das Objekt, welches die Informationen des Sensors zusammenfasst, heißt SimulatorModelRadar. Dieses enthält 19 Attribute, die die Eigenschaften des simulierten Sensors beschreiben. Zusätzlich gibt es noch Beziehungen zu zehn Klassen, die die Daten des Sensors speichern. 

Die Klassen, die in dieser Arbeit verwendet werden, sind:

\begin{itemize}
    \item SimulatorBitNode: Der SimulatorBitNode ist eine Baumdatenstruktur in der Bitfehler abgelegt werden. Das SimulatorModelRadar besitzt höchstens einen SimulatorBitNode.
    \item SimulatorModelEquipment: Der SimulatorModelRadar kann beliebig viele   \\
    SimulatorModelEquipment-Objekte des Sensors gespeichern. Weil die PNU ein Zubehör des Sensors ist, werden dessen Informationen als SimulatorModelEquipment gespeichert. Dafür gibt es eine Klasse, welche SimulatorModelEquipment erweitert. Diese heißt SimulatorModelPNU.
    \item SimulatorModelSector: Um Sektor Informationen des Sensors zu speichern, gibt es die Klasse SimulatorModelSector. Die Klasse SimulatorModelRadar hat höchstens einen aktuellen Sektor und beliebig viele weitere Sektoren.    
\end{itemize}

Was bei der Betrachtung des grafischen Modells auffällt ist, dass es drei vom SimulatorModelRadar unabhängige Klassen gibt. Diese Klassen sind:

\begin{itemize}
    \item \texttt{Scenario}: Die Klasse Scenario beinhaltet beliebig viele ScenarioTargets, welche beliebig viele Waypoints haben. Man erkennt daran gut wie das Scenario aufgebaut ist. Die ScenarioTargets sind mögliche Objekt, die der Sensor detektieren kann. Das sind zum Beispiel Fußgänger oder PKWs. Diese Objekte bewegen sich auf einem Pfad, welcher durch die Waypoints (auf Deutsch Wegpunkte) definiert wird.
    \item \texttt{TargetSimulation}:  Die TargetSimulation beschreibt die Eigenschaften, wie z.B. Position, Klassifizierung und Radarquerschnitt, eines Radarziels zu einem bestimmten Zeitpunkt. Eine TargetSimulation entsteht, wenn im UI ein Scenario abgespielt wird. Dabei wird die aktuelle Position eines ScenarioTargets, anhand der Wegpunkte berechnet und daraus wird eine TargetSimulation erstellt.
    \item SimulatorModelTrack: Ein SimulatorModelTrack beschreibt ebenso, wie die TargetSimulation, die Eigenschaften eines Radarziels. Der entscheidende Unterschied ist, dass dieses Radarziel nun abhängig vom Sensor ist. Deswegen hat dieser weitere Attribute, welche die Venus benötigt um das Ziel darzustellen. In Tabelle \ref{table:1} werden die entscheidenden Unterschiede der beiden Klassen aufgezeigt.    
\end{itemize}

\begin{table}[h]
    \begin{tabular}{ |c|c|c| } 
        \hline
         & \texttt{TargetSimulation} & \texttt{SimulatorModelTrack} \\ 
        \hline
        Koordinatensystem & Lat/Long & Polarkoordinaten \\ 
        \hline
        Doppler Speed Informationen & - & x \\ 
        \hline
        
   \end{tabular}
   \caption{Vergleich zwischen\texttt{TargetSimulation} und \texttt{SimulatorModelTrack}}
   \label{table:1}
\end{table}

Wie sinnvoll ist es nun diese Klassen vom Modell zu trennen? Das Scenario wird von der UI Komponente verwendet, um es auf einer Karte darzustellen und um es als XML-Datei zu speichern. Wie man in Abbildung \ref{figure:scenarioview} erkennen kann gehört das Scenario zum ScenarioController und lediglich der ScenarioUpdater ist vo ihm abhängig. Deswegen ist es nicht notwendig das Scenario im Modell zu speichern und es müssen keine Änderungen vorgenommen werden.

\begin{figure}[h!]
    \centering
    \includegraphics[width=1\textwidth]{content/assets/Kapitel3/ScenarioViewRelations.png}
    \caption{Beziehung zwischen SimulatorInstance und ScenarioView}
    \label{figure:scenarioview}
\end{figure}

