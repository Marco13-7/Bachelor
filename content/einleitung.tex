\chapter{Einleitung}

\section{Themenumfeld}
Softwaresysteme sind heutzutage im ständigen Wandel und werden insbesondere, seitdem es die Möglichkeit gibt Updates und Patches über das Internet bereitzustellen, immer langlebiger. Wenn Quellcode eine lange Lebensdauer haben soll, ist es unabdingbar, dass dieser eine gute Wartbarkeit hat. Ansonsten können der Zeitaufwand und die Entwicklungskosten bis ins Unermessliche steigen. Diese Kosten entstehen, wenn Fehlerbehebungen am Code durchgeführt werden müssen oder dieser um funktionelle Features erweitert wird. Ein wichtiges Konzept, um die Wartbarkeit von Softwaresystemen zu verbessern ist eine Modularisierung der verwendeten Softwarekomponenten.

Im Rahmen dieser Bachelorarbeit wird eine Java-Anwendung mit einer modularen Architektur entworfen und implementiert. Um das umzusetzen wird das Framework \Gls{equinox} verwendet, welches die OSGi-Kernspezifikationen implementiert. OSGi bietet ein Konzept zur Modularisierung und Komponentisierung von Java-Applikationen, und zwar nicht nur zur Entwicklungszeit, sondern auch zur Laufzeit~\cite{horn}. Diese Applikation entsteht bei der Firma Thales Deutschland GmbH am Standort in Ditzingen im Bereich LAS (Land and Air Systems) in der Abteilung GSR(Ground- and Surfaceradars). In der Abteilung werden Radarsysteme für den zivilen, wie den militärischen Einsatz entwickelt.  Die Abteilung ist in das Sensorteam und das Venusteam aufgeteilt. Das Venusteam ist für die Steuerungssoftware der Radarsensoren verantwortlich. Die Steuerungssoftware heißt Venus und bietet ein \acrfull{mmi} zur Steuerung von verschiedenen Radargeräten. Zudem visualisiert die \Gls{venus} Sensordetektionen auf einer Karte und liefert weitere relevante Sensorinformationen, wie z.B. Hardwarefehlermeldungen in Tabellen. Die Software läuft auf einem Laptop und ist über einen speziellen Anschluss mit dem Sensor verbunden.  Zusätzlich entwickeln sie weitere Eclipse Applikationen zum Warten und Testen der Radarsensoren. Das Sensorteam entwickelt die Hardware und Software der verschiedenen Sensormodelle.

\section{Aufgabenstellung}

Das Sensorteam will einen Trainingssimulator erstellen, um neuen Benutzern die Funktionalitäten der Venus näher zu bringen. Bei der Trainingssimulation nehmen ein Trainer und ein bis mehrere Trainees teil. Jeder der beteiligten Personen verfügt über ein Laptop mit der entsprechenden Software. Während der Trainingssimulation, gibt der Trainer Simulationsdaten in die Benutzeroberfläche der Trainingssimulator Anwendung ein. Diese Informationen werden über eine Netzwerkschnittstelle an die Computer der Trainees übermittelt und dort in Echtzeit simuliert. Die Aufgabe des Trainingssimulators ist es, dass die Trainees lernen, die simulierten Daten mit der Steuerungssoftware korrekt zu erfassen und richtig interpretieren. Ein detaillierterer Ablauf wird in Kapitel X.X beschrieben.

Das Ziel dieser Bachelorarbeit ist es, die beschriebene Trainingssimulator Anwendung zu implementieren. Um Wiederverwendbarkeit der Komponenten und eine einfache Erweiterung um funktionelle Features zu gewährleisten, steht eine modulare Architektur der Software im Vordergrund. Die Grundlage für diese Anwendung ist ein Radarsimulator, welcher einen realen Radarsensor simuliert. Dieser Radarsimulator wurde von früheren Praktikanten und Werkstudenten programmiert und wird derzeitig für interne Tests der Venus verwendet. Jedoch tendiert die Softwarearchitektur des Radarsimulators zu einer monolithischen Architektur, da einzelne Softwaremodule mehrere verschiedene Funktionen haben und wenig klare Schnittstellen zwischen den bereits vorhandenen Modulen definiert sind. Damit darauf sinnvoll aufgebaut werden kann, muss diese Architektur im Vorhinein überarbeitet werden. Zudem soll die Anwendung so angepasst werden, dass der Großteil der Softwarekomponenten in der Testanwendung und in der Trainingssimulator Anwendung verwendet werden können. Dazu muss eine entsprechende Architektur erstellt werden, die durch möglichst wenige Änderungen zu den verschiedenen Ausprägungen abgeleitet werden kann.

Eine weitere Aufgabe ist es, die Trainingsanwendung, sowie die Testanwendung und um drei Features ergänzt werden. Diese Features sind \acrfull{stt}, die Bereitstellung des Dopplertons während \acrshort{stt} und die Einbindung der Detektionswahrscheinlichkeit von Radarobjekten. Single Target Tracking ist die Funktion ein Radarziel über einen bestimmten Zeitraum zu verfolgen. Diese Funktion wird vom Sensor bereitgestellt und nicht von der Venus. Deshalb muss der Simulator diese Funktion bereitstellen. Wenn das Ziel von \acrshort{stt} erfasst wird soll daraufhin der Dopplerton des verfolgten Objektes per Audio abgespielt werden. Wenn Ziele sich am Rande des Detektionsradius befinden, können diese in einem realen Scenario nicht immer detektiert werden. Die Wahrscheinlichkeit, dass diese nicht detektiert werden kann berechnet werden. Der Simulator soll diese Wahrscheinlichkeit für jedes Ziel auswerten und diese dann nur zu einem gewissen Prozentanteil anzeigen lassen. Diese Features sollen eine realistischere Trainingsumgebung für neue Nutzer des Trainingssimulators bieten.

\newpage

\section{Aufbau der Thesis}
Zu Beginn der Thesis werden die allgemeinen Grundlagen der Konzepte und Technologien, die in dieser Arbeit verwendet werden, erläutert. Als nächstes wird die alte Architektur des Radarsimulator analysiert und auf Schwachstellen überprüft. Zudem werden die funktionellen Anforderrungen der Trainingssimulator- und Testsimulatoranwendung aufgelistet und es wird untersucht, an welchen Komponenten die vorgegebenen Erweiterungen anknüpfen müssen. Nach der Analyse wird das Konzept der neuen Architektur vorgestellt. Ein besonderes Augenmerk wird dabei auf die Softwaremodule und deren Schnittstellen gelegt. Des Weiteren werden den verschiedenen Anwendungsausprägungen die Softwaremodule zugeordnet und genauer erklärt. Außerdem wird gezeigt, wie das Datenmodell verändert wurde und von den neuen Softwarekomponenten verwendet wird. Ergänzend wird darauf eingegangen, wie die \Gls{buildpipeline} die Testanwendung, die Traineranwendung und die Traineeanwendung baut und testet. Im folgenden Kapitel wird erklärt, wie die Softwarekomponenten implementiert und zusammengesetzt wurden. Danach wird bewertet, wie der Status der Applikation am Ende des Projektes aussieht, bzw. ob alle wichtigen Funktionen enthalten sind und die Anwendung stabil läuft. Zuletzt gibt es noch eine Schlussbetrachtung der Thesis, in der Erkenntnisse der Arbeit erläutert werden und wie es in der Zukunft mit der Anwendung und den Technologien weitergehen soll.
